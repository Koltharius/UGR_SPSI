\documentclass[10pt,a4paper,spanish]{report}

\usepackage[spanish]{babel}
\usepackage[utf8]{inputenc}
\usepackage{amsmath, amsthm}
\usepackage{amsfonts, amssymb, latexsym}
\usepackage{enumerate}
\usepackage[official]{eurosym}
\usepackage{graphicx}
\usepackage[usenames, dvipsnames]{color}
\usepackage{colortbl}
\usepackage{multirow}
\usepackage{fancyhdr}
\usepackage[all]{xy}
\usepackage{pgfplots}
\usepackage{algpseudocode}
\usepackage{listings}
\usepackage{titlesec}
\usepackage{minted}

\pgfplotsset{compat=1.5}

% a4large.sty -- fill an A4 (210mm x 297mm) page
% Note: 1 inch = 25.4 mm = 72.27 pt
%       1 pt = 3.5 mm (approx)

% vertical page layout -- one inch margin top and bottom
\topmargin      0 mm    % top margin less 1 inch
\headheight     0 mm    % height of box containing the head
\headsep       10 mm    % space between the head and the body of the page
\textheight   250 mm
\footskip      14 mm    % distance from bottom of body to bottom of foot

% horizontal page layout -- one inch margin each side
%\oddsidemargin    0   mm    % inner margin less one inch on odd pages
%\evensidemargin   0   mm    % inner margin less one inch on even pages
%\textwidth      159.2 mm    % normal width of text on page

\usepackage[math]{iwona}
\usepackage[T1]{fontenc}
\usepackage{inconsolata}

\usepackage[pdftex, bookmarks=true,
	bookmarksnumbered=false, % true means bookmarks in
	% left window are numbered
	bookmarksopen=false,     % true means only level 1
	% are displayed.
	colorlinks=true,
linkcolor=webblue]{hyperref}

\definecolor{webgreen}{rgb}{0, 0.5, 0} % less intense green
\definecolor{webblue}{rgb}{0, 0, 0.5}  % less intense blue
\definecolor{webred}{rgb}{0.5, 0, 0}   % less intense red
\definecolor{dblackcolor}{rgb}{0.0,0.0,0.0}
\definecolor{dbluecolor}{rgb}{.01,.02,0.7}
\definecolor{dredcolor}{rgb}{0.8,0,0}
\definecolor{dgraycolor}{rgb}{0.30,0.3,0.30}

\newcommand{\HRule}{\rule{\linewidth}{0.5mm}} % regla horizontal para  el titulo

\pagestyle{fancy}
%con esto nos aseguramos de que las cabeceras de capítulo y de sección vayan en minúsculas

\renewcommand{\chaptermark}[1]{%
	\markboth{#1}{}}
\renewcommand{\sectionmark}[1]{%
	\markright{\thesection\ #1}}
\fancyhf{} %borra cabecera y pie actuales
\fancyhead[LREO]{\bfseries\thepage}
\fancyhead[LO]{\bfseries\leftmark}
\renewcommand{\headrulewidth}{0.5pt}
\renewcommand{\footrulewidth}{0pt}
\addtolength{\headheight}{0.5pt} %espacio para la raya
\fancypagestyle{plain}{%
	\fancyhead{} %elimina cabeceras en páginas "plain"
	\renewcommand{\headrulewidth}{0pt} %así como la raya
}

%%%%% Para cambiar el tipo de letra en el título de la sección %%%%%%%%%%%
\usepackage{sectsty}
\chapterfont{\fontfamily{pag}\selectfont} %% for chapter if you want
\sectionfont{\fontfamily{pag}\selectfont}
\subsectionfont{\fontfamily{pag}\selectfont}
\subsubsectionfont{\fontfamily{pag}\selectfont}
\titleformat{\chapter}{\normalfont\Huge}{}{0pt}{\Huge} % Capítulos sin "Capítulo x" encima del título

\renewcommand{\labelenumi}{\arabic{enumi}. }
\renewcommand{\labelenumii}{\labelenumi\alph{enumii}) }
\renewcommand{\labelenumiii}{\labelenumii\roman{enumiii}: }


\newmintedfile[mypython]{python}{
	frame=lines,
	framesep=2mm,
	baselinestretch=1.2,
	bgcolor=LightGray,
	fontsize=\footnotesize,
	linenos
}

\title{Seguridad y Protección de Sistemas Informáticos \\
Certificados Digitales}
\author{David Sánchez Jiménez}

\begin{document}
\begin{titlepage}
 \begin{center}
  \HRule \\[0.8cm]
  \textsc{\huge Seguridad y Protección \\ de Sistemas Informáticos \\[0.5cm] Puzles Hash}\\[1.6cm]
  \HRule \\[1cm]
  \begin{flushleft}
   \emph{Hecho por:}\\
   David Sánchez Jiménez
  \end{flushleft}
  \vspace{12cm}
  \large{\today}\\
  \vspace{0.5cm}
  \htmladdnormallink{\includegraphics[width=2cm]{Imagenes/88x31.png}}
  {http://creativecommons.org/licenses/by-nc/4.0/}\\[0.5cm]
  \texttt{Prácticas de Seguridad y Protección de Sistemas Informáticos\\ by
   \href{mailto:dasaji92@gmail.com}{David Sánchez Jiménez} is licensed under a \htmladdnormallink{Creative Commons Reconocimiento-NoComercial-CompartirIgual 4.0 Internacional License}
   {http://creativecommons.org/licenses/by-nc/4.0/}}.\\[3mm]
 \end{center}
\end{titlepage}

\tableofcontents
\newpage

% ----------------------------------------------------------------
\chapter{Ejercicio 1}

\section{Enunciado}
\noindent
Elegid una función hash H con salida de n-bits con n $\leq$ 256. Para la función H, realizad, en el lenguaje de programación que queráis, una función que tome como entrada un texto y un número de bits b. Creará un id que concatene una cadena aleatoria de n bits con el texto. Pegará a ese id cadenas aleatorias x de n bits hasta lograr que H(id||x) tenga sus primeros b bits a cero. La salida será el id, la cadena x que haya proporcionado el hash requerido, el valor del hash y el número de intentos llevados a cabo hasta encontrar el valor x apropiado.

\section{Respuesta}
\noindent

\begin{minted}[linenos,numbersep=5pt,gobble=0,frame=lines,framesep=2mm,]{python}
	import formatter
	import hashlib
	import random
	import sys
	from itertools import groupby


	def rand_bits_number():
	    return str(random.getrandbits(256))


	def concatenate(text, number_string):
	    return number_string + text


	def hash_puzzle(concatenation):
	    return hashlib.sha256(concatenation.encode('utf-8 ')).hexdigest()


	def hex_to_bin(hexadecimal):
	    return bin(int('1' + hexadecimal, 16))[3:]


	def print_results(identification, x, h, b, steps):
	    print("\nID: ", identification, "\nX: ",
	          x, "\nHash: ", h, "\nBinary: ", b, "\nSteps: ", steps)


	def puzzle(text, num_bits):
	    condition = False
	    steps = 0
	    number_string = rand_bits_number()
	    identification = concatenate(text, number_string)

	    while condition != True:
	        x = rand_bits_number()
	        concatenation = concatenate(text, x)
	        h = hash_puzzle(concatenation)
	        b = hex_to_bin(h)

	        if len(b.split("1", 1)[0]) == num_bits:
	            steps += 1
	            condition = True
	        else:
	            steps += 1

	    print_results(identification, x, h, b, steps)


	if __name__ == '__main__':
	    text = open(sys.argv[1], mode='r', encoding='UTF-8')
	    num_bits = sys.argv[2]
	    puzzle(text.read(), int(num_bits))

\end{minted}

% ----------------------------------------------------------------
\chapter{Ejercicio 2}

\section{Enunciado}
\noindent
Calculad una tabla/gráfica que vaya calculando el número de intentos para cada valor de b. Con el objeto de que los resultados eviten ciertos sesgos, para cada tamaño b realizad el experimento 10 veces y calculad la media del número de intentos.

\section{Respuesta}
\noindent

% ----------------------------------------------------------------
\chapter{Ejercicio 3}

\section{Enunciado}
\noindent
Repetid la función anterior con el siguiente cambio: Se toma un primer valor aleatorio x y se va incrementadno de 1 en 1 hasta obtener el hash requerido.

\section{Respuesta}
\noindent

% ----------------------------------------------------------------
\chapter{Ejercicio 4}

\section{Enunciado}
\noindent
Calculad una nueva tabla/gráfica similar a la obtenida en el punto 2 pero con la función construida en 3.

\section{Respuesta}
\noindent

% ----------------------------------------------------------------
\end{document}
